\chapter{Introduction} \label{Introduction}

In recent years, machine learning (ML) has witnessed an unprecedented increase in its adoption across various domains, ranging from finance and healthcare to autonomous vehicles and natural language processing. As the ML models can have remarkable ability to learn and make predictions from vast amounts of data and the training of these ML models can be expensive and time-consuming, the ML models have become valuable assets for companies. However, with the increasing adoption of machine learning models comes the concern of protecting the intellectual property (IP) associated with these ML models.

Trusted Execution Environments (TEEs) have emerged as a promising solution to address the challenges associated with IP protection of machine learning models. TEEs provide a secure and isolated environment where sensitive computations, such as those involved in using and training machine learning models, can be executed securely, protecting the confidentiality and integrity of the underlying algorithms and data.

The most significant concern that arises is the risk of model theft or unauthorized replication of the machine learning models. The concern arises mainly when ML application needs to be run in an untrusted environment. The untrusted environment can be, for example, an infrastructure of a customer to whom the machine learning application is distributed. Untrusted environment can also be third-party cloud infrastructure. Unauthorized access to the ML models can lead to their replication, reverse engineering, or even the extraction of sensitive information embedded within them.\cite{ipofml}

Trusted Execution Environments offer a solution to address these concerns. By using hardware-based security technologies, such as Intel SGX (Software Guard Extensions), AMD SEV (Secure Encrypted Virtualization) and ARM TrustZone, TEEs create a secure and isolated enclave where sensitive computations can be isolated from the rest of the system. Isolation ensures that even privileged software cannot access or tamper with the sensitive data and algorithms processed inside the enclave.\cite{teeieee}

This thesis explores the potential of TEEs in protecting Intellectual Property associated with machine learning models when the machine learning application needs to be distributed to the client with the ML model bundled within it. Trusted Execution Environment implementation chosen for closer examination is Intel SGX. This thesis tries to answer to the questions on how Intel SGX can be used to protect the IP of the ML models, what aspects need to be considered and what are the limitations, from the perspective of ML model Intellectual Property protection. Application of Intel SGX is explored and existing techniques and frameworks are discussed. This thesis also provides an example implementation of a machine learning application that uses Intel SGX to protect the IP of the ML model. The research questions of this thesis can therefore be formulated as follows:

\begin{questions}[itemindent=2em]
        \item\emph{What is currently known about different TEE implementations and their potential in protecting IP associated with ML models?\label{rq1}}
        \item\emph{How TEEs, especially Intel SGX, can be used to protect the IP associated with ML models when ML application is distributed to end-users?\label{rq2}}
        \item\emph{What issues and limitations should be considered when applying TEEs, especially Intel SGX, for IP protection associated with ML models?\label{rq3}}
\end{questions}

Chapter \ref{problem} discusses the concerns raised in more detail. Even though this thesis focuses on the problem when ML application is distributed with ML model, the scenario when ML application runs inside untrusted cloud infrastructure is briefly discussed. Although this specific topic has not been previously researched, Chapter \ref{prev-res} explores the previous research around this topic. Chapter \ref{ml} gives a brief introduction to the machine learning and explores different types of machine learning and current popular techniques to implement them. Chapter \ref{tees} presents Trusted Execution Environments and the most common use cases of them. Different TEE implementations are introduced, with the focus on Intel SGX, which is discussed in more detail. Chapter \ref{solution} offers a practical solution to the concerns of ML model IP protection. The example implementation that is made as a part of this thesis is presented in this chapter. To practically demonstrate the limitations of Intel SGX, the example implementation is used to conduct performance testing. The results of the performance testing are also presented in this chapter. Chapter \ref{conclusion} concludes this thesis by pulling together the main findings.
